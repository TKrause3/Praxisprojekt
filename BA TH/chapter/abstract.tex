%!TEX root = ../draft.tex
\chapter*{Abstract}
Bei der Klassifizierung von künstlichen neuronalen Netzen, kann es passieren, dass es zu Fehlern bei der Identifizierung von Objekten kommt. Das liegt häufig daran, dass nicht jede Farbe trainiert wurde, welche bei unterschiedlichen Beleuchtungen auftreten kann. \\\\
Das Ziel dieser Arbeit ist es, zu überprüfen, ob Farbnormalisierung von Datensätzen einen Positiven Einfluss auf die Klassifizierung mittels künstlicher neuronaler Netze hat, und welche Beschränkungen diese haben.\\\\
Um dieser Frage nachzugehen, werden zunächst drei unterschiedliche Normalisierungsverfahren vorgestellt. Anschließend werden vier verschiedene Datensätze mit diesen Verfahren Normalisiert. Erläuterung der theoretischen Grundlagen folgt die Darstellung der Implementierung der Software unter Verwendung des OpenSource Frameworks Tensorflow.Abschließend kommt es zur Durchführung des Trainings der verschiedenen Netze, welche daraufhin verglichen und ausgewertet werden. In einer Diskussion wird nun das Ergebnis der Auswertung anhand der angeführten These evaluiert und ein Fazit der Arbeit getroffen.
