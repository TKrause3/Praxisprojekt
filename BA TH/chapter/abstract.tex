%!TEX root = ../draft.tex
\chapter*{Abstract}
Bei der Klassifizierung von künstlichen neuronalen Netzen kann es passieren, dass es zu Fehlern bei der Identifizierung von Objekten kommt. Das liegt häufig daran, dass nicht jede Farbe trainiert wurde, welche bei unterschiedlichen Beleuchtungen auftreten kann.\\\\
Das Ziel dieser Arbeit ist es, zu überprüfen, ob Farbnormalisierung von Datensätzen die Klassifizierung mittels künstlicher neuronaler Netze positiv beeinflusst, und welche Beschränkungen diese hat.\\\\
Um dieser Frage nachzugehen, werden zunächst die theoretischen Grundlagen zum digitalen Bild und der neuronalen Netze erläutert. Anschließend  werden  verschiedene Verfahren zur Farbnormalisierung vorgestellt und auf ihre Eignung zur Lösung der gegebenen Problemstellung hin untersucht. Es erfolgt die Darstellung der Implementierung der Software unter Verwendung des OpenSource Frameworks Tensorflow. Nachfolgend werden vier verschiedene Datensätze mit den unterschiedlichen Normaliesierungsverfahren bearbeitet. Im Anschluss kommt es zur Durchführung des Trainings der verschiedenen Netze, wobei je ein Netz pro Normalisierungsverfahren traininert wird. Diese Netze werden daraufhin verglichen und ausgewertet. In der Diskussion werden die Ergebnisse evaluiert und ein Fazit der Arbeit gezogen.
