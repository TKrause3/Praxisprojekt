%!TEX root = ../draft.tex
\chapter*{Abstract}
Bei der Klassifizierung von künstlichen neuronalen Netzen, kann es passieren, dass es zu Fehlern bei der Identifizierung von Objekten kommt. Das liegt häufig daran, dass nicht jede Farbe trainiert wurde, welche bei unterschiedlichen Beleuchtungen auftreten kann. Ein möglicher Ansatz war es, die Farben der Datensätze zu normalisieren und dadurch konstanter zu machen. Für dieses Vorhaben gibt es einige Ansätze, welche dieses Problem lösen könnten.\\\\
Das Ziel dieser Arbeit ist es, zu überprüfen, ob Farbnormalisierung von Datensätzen einen Positiven Einfluss auf die Klassifizierung mittels künstlicher neuronaler Netze hat, und welche Beschränkungen diese haben. Dazu wird folgende Forschungsfrage gestellt: Welchen Einfluss, hat Farbnormalisierung von Datensätzen auf die Klassifizierung von künstlichen neuronalen Netzen? \\\\
Um dieser Frage nachzugehen, wurden insgesamt 4 Datensätze verwendet und mittels drei verschiedener Normalisierungs-verfahren überprüft. Dabei ist aufgefallen, dass ein Großteil dieser Verfahren nicht optimal für die Nutzung allgemeiner Datensätze geeignet sind, sondern eher für Datensätze, welche eine einheitliche Umgebung haben. Die Ergebnisse zeigen, dass Farbnormalisierung in einem einheitlichen Datensatz eine Erhöhung der Genauigkeit zu folge haben kann. Auf dieser Grundlage ist es empfehlenswert, zu differenzieren, welche Eigenschaften der Normalisierungsverfahren benötigt werden und zu welchem Zweck der Datensatz erfüllen soll.

