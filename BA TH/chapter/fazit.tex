%!TEX root = ../draft.tex
\chapter{Fazit}\label{s.fazit}
Für ein abschließendes Fazit wird für jedes Normalisierungsverfahren erklärt in welchen Datensätzen diese verwendet werden können und wo die Schwächen liegen.
\section{Gray-World-Algorithmus}
Der Gray-World-Algorithmus kann prinzipiell bei jedem Datensatz angewendet werden. Der Vorteil ist dann erkennbar, wenn verschiedene Lichtfarben in der späteren Erkennung und im Datensatz auftauchen können. Durch diese Methode wird der Kontrast und Dynamik nicht verändert. Das Farbbild wird insgesamt auf Grundlage des Grün-Kanals verschoben. Da dieses Verfahren relativ simpel arbeitet, wird kein großer Einfluss auf spätere künstliche neuronale Netze erzielt.
\section{Histogramm Ausgleich}
Der Histogramm Ausgleich arbeitet auf Grundlage des Histogramms und versucht den Kontrast in dem Bild zu erhöhen. Eingesetzt werden kann es bei Datensätzen, welche einen einheitlichen Hintergrund verwenden, da für die Normalisierung die Farbverteilung des gesamten Bildes verwendet wird. Denn je nach Farbverteilung fällt das Ausgabebild nach dem Ausgleich unterschiedlich aus. Schwächen entstehen bei starken Lichtunterschieden. Sollte ein Bild zu dunkel sein können Farbwerte nicht wiederhergestellt werden.
\section{Histogramm Spezifizierung}
Auch die Histogramm Spezifizierung ist nicht für allgemeine Datensätze geeignet, sondern bei Datensätzen mit einheitlichem Hintergrund. Das liegt zum einen an der Verarbeitung des Histogramms zum anderen an der Tatsache das für dieses Verfahren ein Referenzbild benötigt wird, auf welchem der Datensatz normalisiert wird. Dieses Referenzbild ist Effektiver, desto ähnlicher es dem Datensatz ist. Die Ergebnisse dieses Verfahrens, haben beim Obst Datensatz eine deutlich erhöhte Genauigkeit erzielt. Bei diesem muss aber beachtet werden, das die Laufzeit die längste war und eine gewisse Arbeit voraussetzt.\\\\
Die Verfahren wurden in dieser Arbeit mit vortrainierten neuronalen Netzen überprüft. Es wäre eine interessante frage welchen Einfluss diese Verfahren bei Entstehung der neuronalen Netze zu beobachten und welche Unterschiede dabei auftreten. Außerdem können für den Versuch mit einheitlichen Trainingsbildern größere Datensätze generiert und getestet werden. Im Rahmen dieser Bachelorarbeit konnte nur noch Trainingsdaten in der Mindestgröße erstellt werden. Interessant wäre es, wie sich die Genauigkeiten bei noch mehr Lichtvariationen verhalten würden.\\
Außerdem könnte das verhalten der Normalisierungsmethoden in Kombination miteinander untersucht werden. Beispielsweise wird mit der Histogramm Ausgleichung begonnen und mit einer Spezifikation kombiniert, wie in Kapitel \ref{s.hs} beschrieben. Durch solche Kombinationen könnten sich die unterschiedlichen stärken unterstützen.  