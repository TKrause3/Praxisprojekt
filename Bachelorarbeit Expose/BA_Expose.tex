%%%%%%%%%%%%%%%%%%%%%%%%%%%%%%%%%%%%%%%%%%%%%%%%%%%%%%%%%%%%%%%%%%%%%%%%%%%%%%%
%% LaTeX-Vorlage für Abschlussarbeiten                                       %%
%% (TH Köln -Campus Gummersbach, Fak. 10)                                    %%
%%                                                                           %%
%% Gemäß dem Merkblatt zur Anfertigung von Projekt-, Bachelor-, Master- und  %%
%% Diplomarbeiten der Fakultät 10 von Frau Prof. Dr. Halfmann &              %%
%% Herr Prof. Dr. Rühmann (Version vom 27.01.2008)                           %%
%%                                                                           %%                                                                            
%% Bitte sprechen Sie unbedingt mit Ihrer Betreuerin bzw. Ihrem Betreuer     %%
%% bezüglich der Ausgestaltung Ihrer Arbeit!                                 %%
%%                                                                           %%
%%                                                                           %%
%% MERKKASTEN IN DIESER VORLAGE:                                             %%
%% In dieser Vorlage finden Sie Merkkasten, die Ihnen Informationen          %%
%% zu bestimmten, formalen Aspekten geben. Sprechen Sie immer auch mit       %% 
%% Ihrer Betreuerin bzw. Ihrem Betreuer dazu an.                             %%                       
%% Für die eigene Verwendung der Vorlage entfernen oder kommentieren Sie die %%
%% Merkkasten. Die betreffenden Bereiche für die Merkkasten in der Vorlage   %%
%% sind wie folgt kommentiert: <MERKKASTEN> ... </MERKKASTEN>.               %%                            %%                                                                           %%
%%                                                                           %%
%% LIZENZ:                                                                   %%
%% Diese Vorlage darf nicht kommerziell verbreitet                           %%
%% werden. Eine nicht-kommerzielle Weitergabe ist                            %% 
%% gestattet.                                                                %%
%%                                                                           %%
%% Von Ludger Schönfeld, M. Sc.,
%% 2014-2017                            %%
%%%%%%%%%%%%%%%%%%%%%%%%%%%%%%%%%%%%%%%%%%%%%%%%%%%%%%%%%%%%%%%%%%%%%%%%%%%%%%%

%%%%%%%%%%%%%%%%%%%%%%%%%%%%%%%%%%%%%%%%%%%%%
%% HEADER                                  %%
%%%%%%%%%%%%%%%%%%%%%%%%%%%%%%%%%%%%%%%%%%%%%
\documentclass[a4paper,12pt,oneside]{article}
% Optionen:
% - a4paper => DIN A4-Format
% - 12pt    => Schriftgröße (weitere  
%              grundlegende Fontgrößen: 10pt, 11pt)
% - oneside => Einseitiger Druck

%% Verwendete Pakete:
\usepackage[ngerman]{babel} % für die deutsche Sprache
\usepackage{caption} % Für schönere Bildunterschriften
\usepackage[T1]{fontenc} % Schriftkodierung (Für Sonderzeichen u.a.)
\usepackage[utf8]{inputenc} % Für die direkte Eingabe von Umlauten im Editor u.a.
\usepackage{fancyhdr} % Für Kopf- und Fußzeilen
\usepackage{lscape} % Für Querformat

%% Schriften (Beispiele)
%% Weitere LaTeX-Schriften im "LaTeX Font Catalogue"
%% unter: http://www.tug.dk/FontCatalogue/.
%% ACHTUNG: Ggf. müssen Schriften noch installiert 
%% werden!

% Serifen-Schriften:
\usepackage{lmodern} % Schriftart "Latin Modern"
%\usepackage{garamond} % Schriftart "Garamond"

%Sans Serif-Schriften:
%\usepackage[scaled]{uarial}
%\usepackage[scaled]{helvet}
%%--------------
\usepackage[normalem]{ulem} % Für das Unterstreichen von Text z.B. mit \uline{}
\usepackage[left=3cm,right=2cm,top=1.5cm,bottom=1cm,
textheight=245mm,textwidth=160mm,includeheadfoot,headsep=1cm,
footskip=1cm,headheight=14.599pt]{geometry} % Einrichtung der Seite 

\usepackage{graphicx} % Zum Laden von Graphiken
% INFO: Graphiken einbinden
%
% \includegraphics[scale=1.00]{dateiname}
%
% => Ausgabeformat: PDF-Dokument:
%    Es können die folgenden (Graphik-)formate eingebunden
%    werden: .jpg, .png, .pdf, .mps
% 
% => Ausgabeformat: DVI/PS:
%    Folgende (Graphik-)formate werden unterstützt:
%    .eps, .ps, .bmp, .pict, .pntg
\usepackage{epstopdf}

% Pakete für Tabellen
\usepackage{tabularx} % Einfache Tabellen
\usepackage{longtable} % Tabellen als Gleitobjekte (für die Aufteilung bei langen 
 %Tabellen über mehrere Seiten)
\usepackage{multirow} % Für das Verbinden von Zeilen innerhalb einer Tabelle mit
 % \multirow{anzahl}{*}{Text}

% (Zusatz-)Pakete für Formeln
\usepackage{amsmath}
\usepackage{amsthm}
\usepackage{amsfonts}

\usepackage{setspace} % Paket zum Setzen des Zeilenabstandes
% INFO: Zeilenabstand setzen:
%
% Befehle:
% - \singlespacing  => 1-zeilig (Standard)
% - \onehalfspacing => 1,5-zeilig
% - \doublespacing  => 2-zeilig 
\onehalfspacing % Zeilenabstand auf 1,5-zeilig setzen

% Farbboxen (für die Merkkästen in dieser Vorlage):
\usepackage{tcolorbox}
\tcbset{colback=white,colframe=orange,
        fonttitle=\bfseries}

\usepackage[colorlinks,pdfpagelabels,pdfstartview=FitH,
bookmarksopen=true,bookmarksnumbered=true,linkcolor=black,
plainpages=false,hypertexnames=false,citecolor=black]{hyperref} % Für Verlinkungen
% INFO: Verlinkungen mit dem hyperref-Paket:
%
% Die Angabe von URLs mit dem Befehl \url{} erlaubt einen
% gesonderten Umgang mit Weblinks. Denn die Links werden verlinkt.
% Auch erfolgt automatisch am Zeilenende ein Umbruch des Links.
% Es ist auch nicht erforderlich, Sonderzeichen in der URL manuell zu 
% entschärfen.
%
% TIPP: Sollte ein Umbuch bei einem Link nicht automatisch erfolgen, so kann
% das daran liegen, dass ein/mehrere Zeichen zusätzlich angegeben werden müssen,
% an dem der Link umbrochen werden kann.
% Dies kann mit folgendem Befehl erfolgen (Beispiel):
% \renewcommand*\UrlBreaks{\do-\do_}

% Das Paket "biblatex" für autom. 
% Literaturverzeichnisse:
%\usepackage{csquotes} % Für sprachangepasste Anführungszeichen
%\usepackage[backend=bibtex,style=alphabetic]  
%           {biblatex}
%\addbibresource{bib/literatur.bib}           

%%%%%%%%%%%%%%%%%%%%%%%%%%%%%%%%%%%%%%%%%%%%%
%% DOKUMENT                                %%
%%%%%%%%%%%%%%%%%%%%%%%%%%%%%%%%%%%%%%%%%%%%%
\begin{document}
  
  % Deckblatt
  \pagestyle{empty}
  \begin{titlepage}
    \includegraphics[scale=1.00]{Sources/logo_TH-Koeln_CMYK_22pt}\\
    \begin{center}
      \Large
      Technische Hochschule Köln\\
      Fakultät für Informatik und Ingenieurwissenschaften\\
      \hrule\par\rule{0pt}{2cm} % Horizontale Trennlinie  mit 2 cm Abtand nach unten erzeugen
      \LARGE
      \textsc{Bachelorarbeit - Expose}\\
      \vspace{1cm} % Vertikaler Abstand von 1cm erzeugen
      \huge
      Der Einfluss von Farbnormalisierung in neuronalen Netzen\\
      \vspace{1 cm}
      \large
      Vorgelegt an der TH Köln\\
      Campus Gummersbach\\
      im Studiengang\\
      Medieninformatik\\ 
      \vspace{1.0cm}
      ausgearbeitet von:\\
      \textsc{Torben Krause}\\
      (Matrikelnummer: 11106885)\\
      \vspace{1.5cm}
      \begin{tabular}{ll} % Einfache Tabelle ohne Rahmen, mit 2 Spalten erzeugen
          \textbf{Erster Prüfer:} & Prof. Dr. Martin Eisemann \\
          \textbf{Zweiter Prüfer:} & Prof. Dr. ????? \\
      \end{tabular}
      \vspace{1.5cm}
      \\Gummersbach, im Februar
    \end{center}    
  \end{titlepage}
    
  \newpage
  \pagestyle{plain}
  \pagenumbering{arabic}
  \setcounter{page}{1}
  
  \tableofcontents
  
  \newpage
  
  \section{Problemstellung}
  Das Trainieren eines künstlichen neuronalen Netzes ist ein langwieriger Prozess, in welchem 
  eine große Menge von Ressourcen benötigt werden. Zum einen werden Hardwarekomponenten benötigt, 
  welche für Matrizenmultiplikation geeignet sind, zum anderen wird je nach Umfang des 
  neuronalen Netzes eine gewisse Trainingszeit vorausgesetzt. Zusätzlich werden für ein 
  effektives neuronales Netz oftmals mehrere tausend Trainingsdaten benötigt, welche zusätzlich 
  durch ihre Größe die Trainingszeit beeinflussen. Neben der Größe der Bilder, spielt die  
  Ausleuchtung der Objekte auf den Bildern eine große Rolle. Durch eine einseitige Beleuchtung 
  entstehen Schatten und Erhellungen, welche die Farben der Objekte verändern. Durch 
  die verfälschten Farbwerte, können Problem bei der richtigen Zuordnung der Objekte entstehen. 
  Diese können den Lernprozess des künstlichen neuronalen Netzes bremsen, was dazu führt das die 
  Trainingszeit angehoben wird um alle Eigenschaften der definierten Klassen zu identifizieren.
  \vspace{1 cm}
  \section{Lösungsansatz}
  Eine Möglichkeit um Trainingszeiten optimieren zu können besteht darin, die Farbwerte 
  der Trainingsdaten zu Normalisieren um mögliche Farbverfälschungen, die durch Schatten und 
  Überbelichtungen entstehen auszugleichen. Dafür werden beispielsweise benachbarte Pixel 
  welche im Gleichen Farbraum liegen auf einen Farbwert angepasst, damit keine großen 
  Farbabweichungen auftreten. Es gibt viele derartige Algorithmen, welche unterschiedliche 
  Ansätze verfolgen. Um einen Einblick in die Möglichkeiten der Algorithmen zu zeigen werden 
  einige dieser, im folgenden Beschrieben:\vspace{1 cm}
  \begin{itemize}
  \item Die umfassende Farbnormalisierung versucht, die Lokalisierungs- und 
  Objektklassifizierungsergebnisse in Kombination mit der Farbindizierung zu erhöhen. Dieser 
  Algorithmus ist iterativ und arbeitet in zwei Stufen. Die erste Stufe besteht darin, den 
  roten, grünen und blauen Farbraum mit normalisierter Intensität zu verwenden, um jedes Pixel 
  zu normalisieren. Die zweite stufe sorgt für eine Normalisierung jedes Farbkanal separat, 
  wodurch die Summe der Farbkomponenten einem Drittel der Pixelanzahl entspricht. Durch diese 
  Normalisierung wird die Anzahl der Farbwerte in einem Bild verringert.
  \item Ein weiterer Ansatz ist das einfache Threshholding. Zunächst wird ein Schwellwert der 
  Pixel festgelegt. Ist der wert des Pixels höher als der Schwellwert, so wird ihm ein fester 
  Pixelwert zugeteilt (Weiß). Ist der Pixelwert unter dem Definierten Schwellwert, wird ihm ein 
  gegensätzlicher Wert zugeteilt (Schwarz). Dieses Vorgehen blendet alle Farben welche im Bild zu 
  sehen sind aus und hebt die Kanten der Objekte hervor. Es entsteht ein Schwarz-weiß-Bild, in 
  welchem die Belichtung keinen Einfluss hat.
  \item Die Colormap Normalization wird auf Bilder angewandt welche Colormaps standardmäßig 
  verwenden. Diese ordnet die Farben in der Colormap linear von den Datenwerten Minimum bis 
  Maximum der werte zu. Diese Daten werden von -1 bis +1 zugewiesen. Null ist dabei in der Mitte 
  der Farbkarte und stellt den mittleren Farbwert, also Weiß da.
  \end{itemize}   
  Die aufgeführten Verfahren, stellen nur einen Teil der Algorithmen, welche verwendet werden 
  können da. Für die Bachelorarbeit sollen im Vorhinein einige der Verfahren ausgewählt und 
  getestet werden.\vspace{1 cm}
  \section{Vorgehen}
  Für die Bachelorarbeit sollen verschiedene Bildnormalisierungs-Algorithmen auf einen 
  Datensatz angewendet und im folgenden beim Training verwendet werden. Darauf hin 
  werden die verschiedenen Netze miteinander verglichen. Dabei soll herausgearbeitet werden, 
  welchen Einfluss die normalisierten Bilder auf das Training haben. Bestenfalls, soll die These 
  bestätigt und eine Erhöhung der Effizienz des Trainings erkannt werden. 
  Weiterhin soll geschaut werden welchen Einfluss eine Verringerung der Trainingsdaten auf das 
  neuronale Netz haben. Sollte eine Normalisierung von Daten, die Effizienz des 
  Trainings erhöhen, könnten Ressourcen eingespart werden.\\\\
  \textbf{Geplante Durchführung}\\\\
  (Die Schritte 2-5 werden für jedes ausgewählte Verfahren durchgeführt.)
  \begin{enumerate}
  \item Training eines Datensatzes
  \item Datensatz Farbnormalisieren
  \item Training des Netzes mit dem Normalisierten Datensatz
  \item Auswertung des Trainings
  \item Vergleich des Normalisierungs-Verfahren
  \end{enumerate}
  
  \newpage
  
  \section{Vorläufige Gliederung}
  
  \begin{enumerate} %Gliederung
  \item Deckblatt
  \item Vorwort
  \item Danksagung
  \item Abstract
  \item Inhaltsverzeichnis
  \item Abbildungs- und Tabellenverzeichnis
  \item Abkürzungsverzeichnis
  \item Einleitung
  \begin{enumerate} %Einleitung
  \item Problemstellung und Ziele
  \item Anforderungen
  \item Strucktur der Arbeit
  \end{enumerate} %Einleitung
  \item Theoretische Grundlagen
  \begin{enumerate} %Theoretische Grundlagen
  \item neuronale Netze
  \begin{enumerate} %neuronale Netze
  \item Funktionsweise künstlicher neuronaler Netze
  \item Entstehung von Trainingsdaten
  \end{enumerate} %neuronale Netze
  \item Das digitale Bild
  \begin{enumerate} %Das digitale Bild
  \item Aufbau vom Bildern
  \item Einfluss von Belichtungen
  \item Bild-Normalisierung
  \begin{enumerate}
  \item umfassende Farbnormalisierung 
  \item Threshholding
  \item Colormap Normalization
  \end{enumerate}
  \item Begründung der These
  \end{enumerate} %Das digitale Bild
  \end{enumerate} %Theoretische Grundlagen
  \item Methodik und Durchführung
  \item Auswertung der Ergebnisse
  \item Diskussion
  \item Fazit
  \item Literaturverzeichnis
  \item Anhang
  \item Eidesstattliche Erklärung
  \end{enumerate} \vspace{2 cm} %Gliederung
 
  
  \section{Zeitplan}  
  
  \begin{enumerate}
  \item Ab 11.02     	Vorbereitung und Recherche
  \item Start 11.03: 	Bachelorarbeit
  \item Bis 11.03: 		Literaturrecherche
  \item Bis 01.04: 		Durchführung des praktischen Teils + Auswertung der Trainingsdaten
  \item Bis 15.04:		Rohfassung Hauptteil
  \item Bis 22.04:		Rohfassung Einleitung + Schluss
  \item Bis 29.04:		Überarbeitung + Korrektur
  \item Bis 06.05:		Layout + Titelblatt
  \item Bis 08.05:		Druck + Binden
  \item Bis 10.05:		Abgabe
  \end{enumerate}
  
\end{document}

